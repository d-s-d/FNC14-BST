% IEEE standard conference template; to be used with:
%   spconf.sty  - LaTeX style file, and
%   IEEEbib.bst - IEEE bibliography style file.
% --------------------------------------------------------------------------

\documentclass[letterpaper]{article}
\usepackage{spconf,amsmath,amssymb,graphicx,listings,hyperref,array,float,epstopdf,tikz}
\lstset{basicstyle=\ttfamily\footnotesize,numbers=left,stepnumber=2,frame=single,language=c,captionpos=b}
\usepackage[utf8]{inputenc} % UTF-8 was invented to be used.
\usetikzlibrary{positioning}
\usetikzlibrary{shadows}

% Example definitions.
% --------------------
% nice symbols for real and complex numbers
\newcommand{\R}[0]{\mathbb{R}}
\newcommand{\C}[0]{\mathbb{C}}



% bold paragraph titles
\newcommand{\mypar}[1]{{\bf #1.}}
 
% Title.
% ------
\title{How to write a fast Optimal Binary Search Tree Implementation}
%
% Single address.
% ---------------
\name{Jeremia Bär, Stefan Dietiker}
\address{Department of Computer Science\\ ETH Z\"urich\\Z\"urich, Switzerland}

% For example:
% ------------
%\address{School\\
%		 Department\\
%		 Address}
%
% Two addresses (uncomment and modify for two-address case).
% ----------------------------------------------------------
%\twoauthors
%  {A. Author-one, B. Author-two\sthanks{Thanks to XYZ agency for funding.}}
%		 {School A-B\\
%		 Department A-B\\
%		 Address A-B}
%  {C. Author-three, D. Author-four\sthanks{The fourth author performed the work
%		 while at ...}}
%		 {School C-D\\
%		 Department C-D\\
%		 Address C-D}
%

\DeclareMathOperator{\depth}{depth}

\begin{document}
%\ninept
%
\maketitle
%

\begin{abstract}
In this report our aim is to investigate methods to optimally implement a
dynamic programming algorithm to construct optimal binary search trees. Our
experimental results clearly demonstrate that concise changes to the structure
of the algorithm, such as the order of the loops, the memory layout, etc., boost
its performance significantly. Furthermore, we demonstrate how to apply
vectorization to this algorithm. The best implementation outperforms the
straight-forward baseline implementation by a factor of 6.
\end{abstract}


\section{Introduction}\label{sec:intro}
Given the simple and recursive structure of a binary search tree, it is easy to
deduce an algorithm to find a specific key in the tree. However, as every binary
search tree can be restructured arbitrarily yielding a new valid binary search
tree holding the same set of keys and, in general, there are great variances
among these tree structures concerning the lookup cost, the "naked"
binary search tree must be accompanied by an algorithm, the sophistication of
which lies in its capability to structure the tree optimally with respect to the
lookup cost.

For dynamic binary search trees, various algorithms have been developed to
balance the tree on inseration and deletion (references). In the case of static
binary search trees, on the other hand, the set of keys is given a priori, along
with a probability distribution that indicates the likeliness of an individual
key being searched for. Based upon this, one can construct an optimal binary
search tree, that is, a search tree which has a minimal expected lookup cost.

In the following we introduce a dynamic programming algorithm, which solves the
problem by examining combinations of optimal subtrees. Our objective then is to
optimally implement this algorithm on a modern Intel-based hardware platform.

Our interest in the algorithm emerges from the fact that the algorithm, in its
bare form, has a straight-forward design which makes its properties apparant:
its runtime is $O(n^3)$, whereas the total memory being accessed is $O(n^2)$.

\section{Background}

As a group we worked on the static optimal binary search tree problem.
In this section we give a mathematical description of the optimal binary
search tree problem. The algorithm implemented and optimized to solve the
problem is presented and a cost measure introduced. The presentation is
based on the algorithms book by Cormen et al.~\cite{MITBook}. We conclude
by giving a short overview of two asymptotically faster algorithms.

\mypar{Problem Statement} Let $K = \{k_1, k_2, \dots, k_n\}$ be a sequence
of distinct ordered keys. For each key $k_i$, let $p_i$ be the probability
that a given search is for key $k_i$. Such a search is called successfull.
Furhter, let $D = \{d_0, d_1, \dots, d_n\}$ be the set of dummy keys
returned for unsuccessfull searches as follows: $d_i$ represents searches
for values between $k_i$ and $k_{i+1}$, $d_0$ represent the values smaller
$k_0$ and $d_n$ the values larger $k_n$. For each dummy key $d_i$, let
$q_i$ be the probability that a given search returns $d_i$.
A valid solution to the static binary search tree problem is any binary
search tree $T$ that has $K$ as nodes and $D$ as leaves.
The cost of a search in a such a tree $T$ is defined as the depth of the key
found plus one. Since $K$ and $D$ cover all possible searches,
$\sum_{i=1}^n p_i + \sum_{j=0}^n q_j = 1$ and we can compute the expected
cost of a search in $T$ as:
\begin{align}
  %\mathbb{E}[\text{search cost in } T] =
  %\nonumber\\
  %\mathcal{E}_T &=
  \sum_{i=1}^n (\depth_T(k_i) + 1) \cdot p_i
   + \sum_{i=0}^n (\depth_T(d_i) + 1) \cdot q_i
  \nonumber\\
  % I just can't fit the formula to width :-(
  = 1 + \sum_{i=1}^n \depth_T(k_i) \cdot p_i
      + \sum_{i=0}^n \depth_T(d_i) \cdot q_i
  \label{eqn:cost}
\end{align}
A static binary search tree is called optimal if its expected search cost
is minimal amongst all valid solution trees.
We can now formulate the static optimal binary search tree problem as
follows: Given $K$, $P$ and $Q$, find the optimal binary search tree.

\mypar{Algorithm} Devising an algorithm to solve the problem requires a
crucial insight on the problem structure: Given an optimal tree $T$ for
keys $k_1$ to $k_n$ with root $k_r$, its left subtree $T_L$ is an optimal
solution for the keys $k_1$ to $k_{r-1}$. Clearly, it has to be a binary
search tree for these keys. Then, if it were not optimal, one could replace
$T_L$ by an optimal binary search tree for the keys $k_1,\dots,k_{r-1}$
obtaining a better solution $T'$ for the original problem. However, this
contradicts the optimality of $T$ and hence $T_L$ is optimal as well. This
argument holds anologous for all subtrees in $T$.

From this insight we can construct an algorithm. Let $e[i,j]$ denote the
expected search cost in the optimal binary search tree continaing keys
$k_i$ through $k_j$. If such a tree is used as a left or right subtree to
construct a tree containing more keys, the depth of its nodes increases by
one. Following \autoref{eqn:cost}, the subtree's expected search cost
increases by
\begin{align}
  w(i,j) &= \sum_{l=1}^j p_l + \sum_{l=i-1}^j q_l \nonumber\\
         &= w(i,r-1) + p_r + w(r+1,j)
  \label{eqn:w}
\end{align}
where the second expression reflects the recursive structure of the problem
again. This allows to express the search cost of a binary search tree over
$k_i$ to $k_j$ with root $k_r$ constructed from its subtrees as
\begin{align}
  e[i,j] = e[i,r-1] + e[r+1,j] + w(i,j)
  \label{eqn:e-intermediate}
\end{align}
Knowing the optimal expected search costs for all possible subtrees, we can
construct the optimal binary search tree by choosing the key as root that
minimizes \autoref{eqn:e-intermediate}, i.e
\footnote{Note that for simplicity, we do not include border cases here.
For a complete discussion of the algorithm and its mathematics refer to
\cite{MITBook}.}
\begin{align}
  e[i,j] = \min_{i\leq r\leq j} \{e[i,r-1] + e[r+1,j] + w(i,j)\}
  \label{eqn:e}
\end{align}
This expression can now direclty be translated into code, as is shown in
\autoref{lst:baseline}. The cost of the subtrees is computed using dynamic
programming. The table \texttt{e[IDX(i,j)]} is used to store the
expected search cost for the optimal search tree covering keys in $k_i$
through $k_j$\footnote{Note that the code actually uses zero-indexing.}. We
fill the table \texttt{e} diagonal by diagonal. This corresponds to
computing first all the subrees continaing one key, then containing two
keys and so forth, as indicated by the length variable \texttt{l}. The
innermost \texttt{r}-loop iterates over all valid roots for the subtree to
find the optimal binary search tree for the current keys. We refer to the
expected cost \texttt{e[IDX(i,j)]} to be computed as the \emph{target
cell}. A second table called \texttt{root} is used to keep track which root
was chosen for the current subtree to be able to reconstruct the overall
optimal tree in the end. In the code section, we have omitted the
initialization code. The computation is dominated by the triple loop shown.
\begin{lstlisting}[
  caption={Basline Implementation},
  label=lst:baseline,
  float
]
for (l = 1; l < n+1; l++)
  for (i = 0; i < n-l+1; i++)
    j = i+l;
    e[IDX(i,j)] = INFINITY;
    w[IDX(i,j)] = w[IDX(i,j-1)] + p[j-1] + q[j];
    for (r = i; r < j; r++ )
      t = e[IDX(i,r)] + e[IDX(r+1,j)]
          + w[IDX(i,j)];
      if (t < e[IDX(i,j)])
        e[IDX(i,j)]    = t;
        root[IDX(i,j)] = r;
\end{lstlisting}

\mypar{Cost Analysis} The presented algorithm has runtime $O(n^3)$. As can
be seen in \autoref{lst:baseline}, the computation involves additions and
comparisions only. We define the cost function of the algorithm as the
number of floating point additions and floating point comparisons:
\begin{align*}
C(n) = (\#\text{adds}(n), \#\text{comps}(n))
\end{align*}
The body of the second loop is executed a total of
\begin{align*}
\sum_{i=1}^{n} i = \frac{n(n+1)}{2}
\end{align*}
times. The body of the innermost loop is executed a total of
\begin{align*}
\sum_{i=1}^{n} l\cdot(n-l) = \frac{1}{6}(n^3-n)
\end{align*}
times.  Hence, the cost function is defined as:
\begin{align*}
C(n) = \left(\frac{1}{3}(n^3+3n^2+2n), \frac{1}{6}(n^3 - n)\right)
\end{align*}

\mypar{Alternative Algorithms} The presented algorithm was originally
published by Knuth. He provides further analysis yielding that under
certain conditions the root will not change, allowing to neglect some of
the \texttt{r}-iterations of the algorithm. The final algorithm has runtime
$O(n^2)$. Details are found in \cite{Knuth70}. If an approximation of the
result is sufficient, Mehlhorn \cite{Mehlhorn75} showed that balancing
probabilities in the left and right subtrees already yields results close
to optimal.

\section{Optimization}
\mypar{Turn Columns into Rows} Examining the innermost loop, we see that the
the table is accessed row- and column-wise. Thus, the memory is accessed in
strides of $1$ and $n+1$, respectively. However, with two simple changes to the
algorithm, we can avoid strides of $n+1$ and replace them with accesses of
stride $1$: We make use of the lower half of the square table in that we store
newly calculated values not only at $(i,j)$ but also at $(j,i)$. That is,
instead of accessing a column in the upper half, we access a row in the
lower-half. This needs only two minor changes to the algorithm as described in
listing \ref{lst:baseline}: First, line $7$ turns into
\begin{center}
\verb:t = e[IDX(i,r)] + e[IDX(j,r+1)]:, 
\end{center}
and after line $10$, we would insert 
\begin{center}
	\verb:e[IDX(j,i)] = t;:.
\end{center}
\mypar{Bottom-Up} The order in which individual values are calculated is along
the diagonals, as visualized in Figure TODO. We notice that, for two distinct
cells on a diagonal, their corresponding rows and colums intersect exactly in
one cell. By changing the outermost two loops such that the table is being built
up row-wise and bottom-up (as visualized in Figure TODO) we can thus improve the
tomporal locality with respect to the accesses to the row that is currently
being built-up.

\mypar{Partial Results} We can further increase the temporal locality by
swapping the innermost two loops. Conceptually speaking, instead of calculating
the minimum value over a full row and column, we store intermediate values for
an entire row. That is, we take the first value, let it be $q_i$, add it to
all the values stored in the row $i+1$, compare it to the .

\mypar{Compressed memory layout}
\mypar{Blocking}
\mypar{Vectorization}
\mypar{Scheduling}
\mypar{Alignment}



\section{Experimental Results}

\begin{table}
\begin{center}
\begin{tabular}{>{\bfseries}ll}
Manufacturer         & Intel Corp.\\
CPU Name             & Core i7-3520M\\
L1-Cache             & 32 KB\\
L2-Cache             & 256 KB\\
L3-Cache             & 4 MB\\
CPU-core frequency   & 2901 MHz\\
Scalar FP-Add Cycles/issue  & 1\\
Vect. FP-Add Cycles/issue (SSE) & 2\\
Vect. FP-Add Cycles/issue (AVX) & 4\\
\end{tabular}
\end{center}
\caption{Processor details.}
\label{tab:processor}
\end{table}

This section assesses the performance increase in the algorithm based on
the methods described in the previous section. The test machine and
software tools used are presented.

\mypar{Measurement Setup}
Measurements were performed on a Lenovo ThinkPad T430s with 8 GiB RAM and
the processor shown in \autoref{tab:processor}. The operating system was
Kubuntu 14.04 64-bit. The compiler used was the GNU C Compiler 4.8.2 with
the flags:
\begin{verbatim}
-O3 -m64 -march=corei7-avx -mavx
  -fno-tree-vectorize
\end{verbatim}

Peak performance is equal to the floating point additions cycles/issue of
the processor, see \autoref{tab:processor}, since our algorithm does not
perform any other floating point operations. That is, 1 fl/c for scalar
code, 2 fl/c for SSE and 4 fl/c for AVX code.
To asses the performance, libperfmon
4\footnote{\url{http://perfmon2.sourceforge.net/}} was used to measure the
number of cycles. The results were obtained by averaging over 10
consecutive executions for a given problem size.
To obtain the measurement data for the roofline-plots, perfplot was used,
together with libpcm 2.6.

There is a crucial difference between the performance plots and the roofline
plots: For the latter, all floating point operations were counted, including
comparisons, whereas for the former, the number of additions from the cost
analysis was used as a basis.

\begin{figure}[htb]\centering
  \includegraphics[width=\linewidth]{plot_data/scalar_performance.png}
  \caption{Performance of the scalar implementations.}
  \label{fig:perf-scalar}
\end{figure}
\mypar{Scalar Performance}
\autoref{fig:perf-scalar} shows the performance of the different scalar
implementations. Peak performance lies at 1 fl/c. The reference
implementation has a performance of 0.2 fl/c for large input sizes. Using
the \emph{transposed} approach improving spatial locality, we can already
see stable performance for large input sizes that do no longer fit the last
level cache. We can again see the expected performance boost when changing
the access pattern of the table to \emph{bottom-up}, thanks to further
improved spatial locality. With a performance of about 4.5 fl/c, this
corresponds to a performance gain of more than 2x to the baseline. Changing
to the triangular memory layout improves performance to up to 0.5 fl/c. We
assume this is due to the increased performance of the CPU prefetcher.
However, we were not able to verify this. The \emph{blocking} approach,
based on the squared memory layout, gives another performance boost due to
increased reuse in the CPU registers and fewer memory write-backs. The
final scalar performance of our implementation lies at 62\% peak
performance which is equivalent to a performance gain of roughly 3x.

\begin{figure}[htb]\centering
  \includegraphics[width=\linewidth]{plot_data/triangular_vector_performance.png}
  \caption{Vectorized Performance based on \emph{Triangle}.}
  \label{fig:perf-triangular}
\end{figure}
\mypar{Vectorized Performance, triangular}
\autoref{fig:perf-triangular} shows the performance of our SSE and AVX
vector code based on the \emph{Triangle} scalar code. With 0.8 fl/c
performance for the SSE code, the speedup through vectorization lies at
1.6x below the theoretical maximum of 2x. From SSE to AVX, the speedup lies
at 1.2x compared to the expected 2x. Overall, the speedup from the baseline
lies at 4.6x.

\begin{figure}[htb]\centering
  \includegraphics[width=\linewidth]{plot_data/blocked_vector_performance.png}
  \caption{Vectorization based on \emph{Blocking}.}
  \label{fig:perf-blocked}
\end{figure}
\mypar{Vectorized Performance, blocked}
\autoref{fig:perf-blocked} shows the performance of our SSE vector code
based on the \emph{Blocked} approach. The AVX variant performance worse and
is thus omitted. Having a performance of about 1.1 fl/c, the SSE code
yields a 1.8x speedup compared to the best scalar code and a 5.5x speedup
to the baseline implementation. Note that the blocking SSE code performs
better than the triangular AVX code.

\begin{figure}[htb]\centering
  \includegraphics[width=\linewidth]{roofline-data/roofline_triangular.png}
  \caption{Roofline Plot based on \emph{Triangle}.}
  \label{fig:roofline-triangular}
\end{figure}
\begin{figure}[htb]\centering
  \includegraphics[width=\linewidth]{roofline-data/roofline_blocked.png}
  \caption{Roofline Plot based on \emph{Blocked}.}
  \label{fig:roofline-blocked}
\end{figure}
\mypar{Roofline Analysis}
\autoref{fig:roofline-triangular} and \autoref{fig:roofline-blocked} show
the roofline plots of our respective solutions. Note that for the roofline
plots, libpcm was used to measure all floating point operations including
comparisions. Thus, performance numbers are different than in the
performance plots. We can see in \autoref{fig:roofline-blocked}, scalar
performance is almost at the peak of 1 fl/c. Also, the SSE code comes close
to optimal. In contrast to the performance plots using our theoretical flop
count, here the AVX variant performs slightly better than the SSE one. The
blocked scalar and SSE versions are the fastest for their category. For the
\emph{Triangle} case in \autoref{fig:roofline-triangular}, we have the best
overall performance with the AVX variant yielding around 2 fl/c. This
corresponds to an overall speedup achieved of roughly 6x.

\section{Conclusions}

We could demonstrate that with few relatively simple changes to the original
algorithm, one can significantly improve the performance of an implementation.
Since calculating tables by combining existing entries is inherent to dynamic
programming, it is likely that the observations we made can be translated to
other dynamic programming problems as well.

While in the case of the scalar implementations, very little is to gain by
further optimizing the code, as we are close to the computational bound, the
roofline-plots clearly indicate that in the case of vectorization, there is
room for refinement.

\appendix
\section{Platform Details}

Measurements were done on a Lenovo ThinkPad T430s with the following hardware
configuration.

\begin{table}[H]
\begin{center}
\begin{tabular}{>{\bfseries}ll}
Manufacturer         & Intel Corp.\\
CPU Name             & Core i7-3520M\\
L1-Cache             & 32 KB\\
L2-Cache             & 256 KB\\
L3-Cache             & 4 MB\\
No. of cores         & 2\\
CPU-core frequency   & 2901 MHz\\
FP-Add Cycles/issue  & 1\\
FP-Mult Cycles/issue & 1\\
Peak Performance     & 5.8 GFLops/s $\hat{=}$ 2 flops/cycle
\end{tabular}
\end{center}
\caption{Solution to Exercise 2.}
\label{tblProc}
\end{table}

The operating system being used is the 64-bit variant of Kubuntu 14.04. The
compiler being used is the GNU C Compiler, version 4.8.2. The following flags
were used for compilation:

\begin{verbatim}
-O3 -m64 -march=corei7-avx -Wall\
  -mavx -fno-tree-vectorize
\end{verbatim}



% References should be produced using the bibtex program from suitable
% BiBTeX files (here: bibl_conf). The IEEEbib.bst bibliography
% style file from IEEE produces unsorted bibliography list.
% -------------------------------------------------------------------------
\bibliographystyle{IEEEbib}
\bibliography{bibl_conf}

\end{document}

