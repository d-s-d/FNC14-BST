
\section{Introduction}\label{sec:intro}
A binary search tree is a simple and easy to understand, recursive data
structure. It solves this and that problem, and is vitaly important for this and
that. 

\[Give informal defintion.\] Every tree that follows this definition is a valid
binary search tree. However, it is easy to see that for the same set of keys,
there are different valid binary search trees and all of these trees have
different characteristics concerning the cost needed to access specific nodes in
the tree. The value of a concrete algorithm thus lies in its ability to choose
the optimal structure of a search tree for a given set of keys.

Optimality is reached if the lookup cost, which is equal to the depth of the
node being searched, is minimal. In the case of dynamic binary search trees, a
number of algorithms have been developed to balance the tree upon insertion and
deletion such that tree remains balanced \[ref AVL, SPLAY, etc.\].

In the case of static binary search trees, which are the subject of this paper,
the set of keys that the tree holds is known in advance. Hence, the tree is
constructed once and only used for lookups subsequently. Additionally, we assume
that, for each key, the probability with which this key is accessed is known a
priori.

